%%%%%%%%%%%%%%%%%%%%%%%%%%%%%%%%%%%%%%%%%%%%%%%%%%%%%%%%%%%%%%%%%%%%%%%%%%%%%%%
% Chapter 2: T�tulo del cap�tulo 2
%%%%%%%%%%%%%%%%%%%%%%%%%%%%%%%%%%%%%%%%%%%%%%%%%%%%%%%%%%%%%%%%%%%%%%%%%%%%%%%

%++++++++++++++++++++++++++++++++++++++++++++++++++++++++++++++++++++++++++++++

En el cap�tulo anterior se ha realizado una introducci�n a la programaci�n funcional con Haskell, y ahora se describir� un m�dulo muy �til en la creaci�n del int�rprete, Parsec.

Parsec es un m�dulo de Haskell, un conjunto de funciones exportables que suelen tener una finalidad com�n y se pueden importar en otros programas. En nuestro int�rprete hemos importado Parsec, pues es el m�dulo utilizado en el tutorial que seguimos, Write Yourself
 a Scheme in 48 hours.

Parsec se dise�� desde cero como una librer�a de parsers con capacidades industriales. Es simple, segura, est� bien documentada, provee de buenos mensajes de error y es r�pida. Se define como un transformador de m�nadas que puede ser apilado sobre m�nadas arbitrarias, y tambi�n es param�trico en el tipo de flujo de entrada. La documentaci�n de la versi�n usada en el presente Trabajo Fin de Grado se puede consultar online en https://hackage.haskell.org/package/parsec-3.1.9

%++++++++++++++++++++++++++++++++++++++++++++++++++++++++++++++++++++++++++++++

\section{Primer apartado de otro capitulo}
\label{:sec1}

