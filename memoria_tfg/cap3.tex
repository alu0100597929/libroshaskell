%%%%%%%%%%%%%%%%%%%%%%%%%%%%%%%%%%%%%%%%%%%%%%%%%%%%%%%%%%%%%%%%%%%%%%%%%%%%%%%
% Chapter 3: Título del capítulo 3
%%%%%%%%%%%%%%%%%%%%%%%%%%%%%%%%%%%%%%%%%%%%%%%%%%%%%%%%%%%%%%%%%%%%%%%%%%%%%%%

%++++++++++++++++++++++++++++++++++++++++++++++++++++++++++++++++++++++++++++++

\section{Introducci\'on a Scheme}
\label{3:sec1}

El lenguaje Scheme es un dialecto de Lisp. Es un lenguaje que, como Lisp, es funcional. Es un lenguaje relativamente sencillo de aprender, con un poco de estudio se puede llegar a tener un conocimiento medio en poco tiempo.\\

Scheme fue elegido para escribir uno de los mejores libros sobre Ciencias de la Computaci\'on que se han escrito: Structure and Interpretation of Computer Programs, de la editorial MIT Press. Asimismo, para mis pruebas he usado el int\'erprete de Scheme del MIT, llamado mit-scheme.\\

El lenguaje usa la notaci\'on polaca para las llamadas a funciones y para cualquier tipo de evaluaci\'on. Adem\'as, cualquier definici\'on o llamada a una funci\'on deber\'a hacerse entre par\'entesis. Asimismo, las listas, contengan lo que contengan, se representan como una lista de ciertos valores de cierto tipo, entre par\'entesis y separados por espacios.\\

Veamos una sesi\'on con el int\'erprete oficial de Scheme para familiarizarnos un poco con el lenguaje:\\

\begin{minipage}{\linewidth}
\begin{small}
\begin{lstlisting}[frame=single]
(+ 3 5)
8
\end{lstlisting}
\end{small}
\end{minipage}

\begin{minipage}{\linewidth}
\begin{small}
\begin{lstlisting}[frame=single]
(sqrt 144)
12
\end{lstlisting}
\end{small}
\end{minipage}

Las funciones se definen mediante la palabra reservada \textbf{define}, una lista que contiene el nombre de la funci\'on y la lista de argumentos (todos separados por espacios), y, por \'ultimo, el cuerpo de la funci\'on. Todo esto debe estar entre par\'entesis.\\

\begin{minipage}{\linewidth}
\begin{small}
\begin{lstlisting}[frame=single]
(define (factorial n)
  (if (= n 0)
    1
    (* n (factorial (- n 1)))))
\end{lstlisting}
\end{small}
\end{minipage}

Las funciones \textbf{car}, \textbf{cdr} y \textbf{cons} act\'uan sobre una lista. La funci\'on \textbf{car} devuelve el primer elemento de la lista (si \'este existe), mientras que \textbf{cdr} devuelve una lista compuesta por el segundo elemento de la lista y todos los que le siguen. La funci\'on \textbf{cons} a\~nade un elemento al principio de una lista (por su izquierda). Se usa un wrapper llamado \textbf{rev} que llama a la funci\'on \textbf{rev-rec} con la lista que se le pasa y la lista vac\'ia \textbf{()}.\\

\begin{minipage}{\linewidth}
\begin{small}
\begin{lstlisting}[frame=single]
(define (rev lst) (rev-rec lst ()))

(define (rev-rec lst acc)
  (cond ((null? lst) acc)
        (else (rev-rec (cdr lst)
                       (cons (car lst) acc)))
        ))
\end{lstlisting}
\end{small}
\end{minipage}

%++++++++++++++++++++++++++++++++++++++++++++++++++++++++++++++++++++++++++++++
\section{Segundo apartado de este cap\'itulo}
\label{3:sec2}

%++++++++++++++++++++++++++++++++++++++++++++++++++++++++++++++++++++++++++++++
\section{Tercer apartado de este cap\'itulo}
\label{3:sec3}