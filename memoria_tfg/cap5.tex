%%%%%%%%%%%%%%%%%%%%%%%%%%%%%%%%%%%%%%%%%%%%%%%%%%%%%%%%%%%%%%%%%%%%%%%%%%%%%
% Chapter 5: Conclusiones y Trabajos Futuros 
%%%%%%%%%%%%%%%%%%%%%%%%%%%%%%%%%%%%%%%%%%%%%%%%%%%%%%%%%%%%%%%%%%%%%%%%%%%%%%%

%++++++++++++++++++++++++++++++++++++++++++++++++++++++++++++++++++++++++++++++

Este cap�tulo es obligatorio.
Toda memoria de Trabajo de Fin de Grado debe incluir unas conclusiones y unas 
l�neas de trabajo futuro 

La programaci�n funcional es un paradigma en auge ahora mismo, sobre todo por sus aplicaciones en paralelismo y matem�ticas, as� como por su capacidad para que los programas sean casi inmunes a ataques de fuzzing, y por dar lugar a programas menos propensos a erorres por parte del programador.

Sin embargo, el aprendizaje de la programaci�n funcional es, sobre todo, lento, y muy relacionado con el �lgebra abstracta, el lambda c�lculo y la teor�a de las categor�as. Es el paradigma m�s abstracto que conozco y requiere una cabeza amueblada para tal fin, por ello lo veo m�s orientado (al menos a priori) a matem�ticos que a ingenieros inform�ticos.

Los programas funcionales son muy, muy modulares, en el sentido de que hay gran cantidad de funciones definidas en ellos (de hecho en la programaci�n funcional no se puede hacer otra cosa), y esto permite que sean poco propensos a errores, pues las funciones no comparten estado, se limitan a recibir entradas y producir salidas (esto se conoce como transparencia referencial o ``pureza''). El programador puede entonces limitarse a pensar c�mo a partir de una entrada producir una salida, y esto lleva en muchos casos a generalizaciones y abstracciones bastante m�s potentes que las que se consiguen en otros lenguajes.

