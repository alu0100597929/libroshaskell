%%%%%%%%%%%%%%%%%%%%%%%%%%%%%%%%%%%%%%%%%%%%%%%%%%%%%%%%%%%%%%%%%%%%%%%%%%%%%
% Chapter 1: Introducci�n 
%%%%%%%%%%%%%%%%%%%%%%%%%%%%%%%%%%%%%%%%%%%%%%%%%%%%%%%%%%%%%%%%%%%%%%%%%%%%%%%

%---------------------------------------------------------------------------------
\section{Secci�n Uno}
\label{1:sec:1}

Todo empez� con un hilo en el foro de internet ``forocoches.com'', en �l hablaban de 
que la programaci�n funcional iba a tener cada d�a m�s relevancia porque cuenta con ventajas de 
las cuales la imperativa carece.\\

A ra�z de ello, me interes� por este paradigma y empec� a (e incluso termin� de) leer numerosos 
libros sobre el lenguaje y la programaci�n funcional en general, y a crear peque�os programas en 
Haskell. Haskell es muy interesante debido a que es el lenguaje con mayor nivel de 
abstracci�n en el que he programado hasta hoy.\\

Haskell es id�neo para crear lenguajes de dominio espec�fico. En otras palabras, antes de 
escribir un compilador se captura el lenguaje a compilar (el lenguaje fuente) en un 
tipo. Las expresiones de ese tipo representar�n t�rminos en el lenguaje fuente y normalmente son 
bastante similares al mismo, a pesar de ser, realmente, tipos de Haskell.\\

Luego se representa el lenguaje objetivo como otro tipo m�s. Finalmente, el compilador es 
realmente una funci�n del tipo fuente al tipo objetivo y las traducciones son f�ciles de escribir y 
leer. Las optimizaciones tambi�n son funciones como cualquier otra (ya que realmente en Haskell 
todo es una funci�n, y adem�s, currificada) que mapean del dominio del lenguaje fuente al codominio
del lenguaje objetivo.\\

Por ello los lenguajes funcionales con sintaxis ligera y un fuerte sistema de tipos se consideran 
muy adecuados para crear compiladores y muchas otras cosas cuya finalidad es la traducci�n.\\

Adem�s, Haskell cuenta con mecanismos de abstracci�n muy fuertes que permiten escribir c�digos 
escuetos que se comportan muy bien, como por ejemplo:\\

\begin{itemize}
  \item reconocimiento de patrones
  \item tipos de datos algebraicos (generalizados o no)
  \item lambdas (y por ello, m�nadas)
  \item plegados de listas
\end{itemize}

A continuaci�n se describen los mencionados constructos; el reconocimiento de patrones y los tipos de datos algebraicos se incluyen en la misma secci�n porque son conceptos que est�n muy relacionados.

%---------------------------------------------------------------------------------
\section{Reconocimiento de patrones y tipos de datos algebraicos}
\label{1:sec:2}



%---------------------------------------------------------------------------------
\section{Lambdas}
\label{1:sec:3}

Bla, bla, bla

%---------------------------------------------------------------------------------
\section{Plegados de listas}
\label{1:sec:4}

Pondremos un ejemplo real codificado por m�, un DFA hecho mediante un plegado de listas por la izquierda.\\

\begin{sourcecode}
  probarDFA :: DFA $->$ [Char] $->$ Bool\\
  probarDFA (DFA i a t) = a . foldl' t i\\
\end{sourcecode}

Un DFA se podr�a implementar en programaci�n imperativa con un bucle for que fuera sobreescribiendo el estado en cada iteraci�n, haciendo un lookup en su tabla de estados dependiendo de su estado actual y el s�mbolo le�do.

Esto, en Haskell, se puede hacer usando la funci�n \textbf{foldl} (aunque aqu� por temas de rendimiento y uso de memoria se ha optado por \textbf{foldl'}).

Se trata de empezar con un acumulador (en este caso, el estado inicial), y nos vamos moviendo por la lista (cadena de entrada) de izquierda a derecha, haciendo un lookup...

%------------------------------------------------------------------------------
\begin{figure}[!th]
\begin{center}
\includegraphics[width=0.5\textwidth]{images/arbolbinario.eps}
\caption{Ejemplo}
\label{fig:ArbolBinario}
\end{center}
\end{figure}
%------------------------------------------------------------------------------

